\chapter{Basic Tutorial}
\label{sec:user_tut}

In this chapter the user will be guided through the creation of a simple Business
Process Diagram, from creating the diagram file to validation and code generation.


%%%%%%%%%%%%%%%%%%%%%%%%%%%%%%%%%%%%%%%%%%%%%%%%%%%%%%%%%%%%%%%%%%%%%%%%%%%%%%%%

\section{Creating a new Business Process Diagram}

These are the basic steps for creating your first Business Process Diagram:

\begin{enumerate}
	
	\item Start Eclipse and select a location for the workspace.  This is where
	all the projects -- BPMN and others -- will be stored.  When starting Eclipse
	for the first time, a welcome screen will be shown.  Read something about the
	features of Eclipse, if you want, then exit the screen.
	
	\item Open the \emph{VSDT} Perspective using the \emph{Window} menu.
	
	\item Select \emph{ New $\rightarrow$ Project $\rightarrow$ General
	$\rightarrow$ Project } in the menu bar.  Open the Navigator View to see the
	newly created Project.
	
	\item On the project, select \emph{New $\rightarrow$ VSDT Meta Diagram} and
	enter a name for the file.  On the last page of the wizard some of the global
	settings for the Business Process Diagram, such as the Title, Description and
	Author name, can be set.  A file with the extension \verb_vsdtd_ is created,
	holding both the semantic model and the notational model (i.e. the layout
	information).
	
	\item By now, the diagram should have opened automatically; otherwise open it
	manually by double-clicking it.  It will be opened with the graphical editor.

\end{enumerate}


%%%%%%%%%%%%%%%%%%%%%%%%%%%%%%%%%%%%%%%%%%%%%%%%%%%%%%%%%%%%%%%%%%%%%%%%%%%%%%%%

\section{Setting up Participants and Business Processes}

With the VSDT, not only individual Business Process Diagrams, but sets of Business
Process Diagrams belonging to the same scenario -- here referred to as Business
Process Systems -- can be modeled.  For this, the modeling starts with defining
the several Participants and the Business Processes they participate in.

\begin{enumerate}

	\item Select a \textbf{Participant} from the palette and click on the canvas.
	A stick-figure will appear.  Repeat for each Participant relevant for the
	Business Process System.  These can be companies, roles, computer systems or
	individual persons.
	
	\item Now select \textbf{Business Process Diagram} from the palette and draw
	it on the canvas.  These represent the individual processes (quite similar to
	'Use cases').
	
	\item Now select the connection form the palette and connect the Participants
	with the Business Processes.
	
	\item Finally, perform a double-click on one of the Business Process Diagram
	nodes, which will open it in a new diagram editor.
	
\end{enumerate}


%%%%%%%%%%%%%%%%%%%%%%%%%%%%%%%%%%%%%%%%%%%%%%%%%%%%%%%%%%%%%%%%%%%%%%%%%%%%%%%%

\section{Modeling a basic Business Process}

Next, we will formulate a simple business process.  Here, we will focus on the
visual elements of BPMN.

\begin{enumerate}

	\item To get started, perform a right-click on the canvas and select
	\emph{Initialize \dots $\rightarrow$ Initialize Pools} and hit \emph{OK} to
	confirm the dialog.  For each Participant associated with the Business Process
	a Pool will be created.
	
	\item Alternatively, select \textbf{Pool} from the top of the palette and
	move the mouse to the canvas.  Press the mouse button and drag it to the lower
	right to create a large Pool.  Enter a name for the Pool and select one of
	the Participants associated with the Business Process using the Properties
	view.
	
	\item Along with the Pool also a Lane will be created.  To create more Lanes,
	select the \textbf{Lane} element from the palette and click on the Pool's
	label (as existing Lanes will fill the Pool's compartment completely).  Note
	that the Lanes can not be moved manually.  According to the BPMN specification
	the first Lane will be invisible (faded out in the editor).
	
	\item Let's create some \textbf{Flow Objects} inside of the Lane.  Select one
	of the Flow Object from the palette, i.e. Events, Activities and Gateways,
	and click inside of the Lane.  In case you selected the Event, a small menu
	will appear, asking whether to create a Start, End, or Intermediate Event;
	otherwise the element will be created right away.
		
	\item Select the \textbf{Sequence Flow} icon from the palette and connect the
	several Flow Objects by pressing the mouse button on the source and dragging
	it to the target.  When connecting the Activity be sure to aim for the label.
	If you hit the Activity's compartment you can not create a connection.  You
	can change the routing style from the tool-bar or add more bend-points to a
	connection by dragging it.
	
	\item Use the Property Sheets to alter the Elements' name, description, type,
	and type-specific attributes.  Select the element, e.g.\ an event, and open
	Eclipse's Property View.  Select the Overview sheet from the tabs to the left
	to find a clearly arranged form holding the various attributes.  If you want
	to set only the type of a Flow Object, e.g. for making an Event a \emph{Message}
	Event, you can also use the element's context menu and select \emph{Edit...
	$\rightarrow$ Set Type}, or use the keyboard shortcut \texttt{Ctrl+T}.

	\item Now select the \textbf{Message Flow} icon from the palette.  Select an
	Activity or an End Event as source and draw the Message Flow to an element in
	a different Pool, or to some point beneath the Pool and select to create a
	new Pool element there.
		
%	\item Finally, we will associate an Activity with a \textbf{Data Object}
%	(however, this will not affect the generated code).  Select the Data Object
%	from the palette and create it on the canvas.  Select the Association connection
%	from the palette and connect the Data Object to the Activity.  Select
%	\emph{BPMN $\rightarrow$ Initialize Input/Output Set} from the Activity's
%	context menu, depending on the associations direction.  Notice the new Input
%	Element in the Activity's property sheet.  This Input Set references all the
%	Activity's incoming/outgoing Data Objects.
	
\end{enumerate}


%%%%%%%%%%%%%%%%%%%%%%%%%%%%%%%%%%%%%%%%%%%%%%%%%%%%%%%%%%%%%%%%%%%%%%%%%%%%%%%%

\section{In-depth Modeling}

Now that we created the diagram visuals, this section deals with the equally
important underlying, non-visual parts of BPMN, such as properties, assignments,
conditions, and service invocations.

\begin{enumerate}

	\item First of all, you have to make sure that the diagram is marked as being
	\emph{executable}, which can be set in the Business Process System's properties
	sheet.  This flag is used for distinguishing process models which are intended
	for code generation from those intended for documentation only.

	\item To define Services and/or Messages, select the \emph{Services...} and
	\emph{Message Channels...} buttons from the Business Process Diagram's property
	sheet.  Alternatively, Services can also be imported using the Web Service
	View, or similar views for other executable languages, like JIAC, which is
	much more comfortable and will be explained in depth later.  Note that depending
	on the target language, Services and Messages can map to different concepts.

	\item Next, we will define the process data, i.e. Properties associated to
	the Pool.  Open the Pool's overview property sheet and click \emph{Process
	Properties...} or select the respective item from the Pool's context menu.
	Create some properties using the buttons in the dialog and edit the values of
	the selected Property using the text fields in the lower part of the dialog.
	Besides the top-level process, Tasks and Subprocesses can hold Properties,
	too, which are available only for that activity or its child activities.
	
	\item Besides ``basic'' data types, such as numbers and strings, you can also
	declare specific complex data types, for example Java classes or ontologies,
	using the \emph{Organize Data Types Dialog}, which is accessible through the
	Business Process Diagram's properties sheet.  These data types can then be
	used as types for Properties.
	
	\item To assign a value to a Property, you have to create an Assignment.
	Open the property sheet of some element in the process and click the
	\emph{Assignments...} button.  Create a new Assignment, select the Property
	and enter an Expression.  Click the button with the dots (\dots) on it to
	open another dialog helping you to enter and validate an expression using
	the VSDT Expression Language VXL (see Appendix~\ref{sec:vxl}).
	
	\item Now that the Properties are declared and have values assigned to them,
	they can be used e.g.\ in condition expressions.  Select a Sequence Flow
	coming from a Gateway (a point where the flow of control branches), set the
	Condition Type to \emph{Expression} and enter the Condition Expression.
	Again, use the button with the three dots(\dots) to validate the Expression.
	
	\item The payload of a Message and the input and output parameters of a
	Service call are represented as properties, too.  To set those, use the
	dialogs for managing messages and services introduced above.
	
	\item To pass the parameter values to the Message (or some Service parameters),
	create one or more Assignments on the Flow Object the Message (or Service) is
	going in or out of.  There are two ways for doing this:
	\begin{itemize}
		\item The easiest way is to use the \emph{Parameter Assignment Dialog}.
		Select the Activity or Event sending or receiving the Message(s) and hit
		the \emph{Parameter Assignments...} button in its property sheet.  The
		dialog will show all of the messages' input and output Properties and
		offer drop-down menus for selecting another Property or entering an
		individual Expression to be assigned to these parameters.
		
		\item For more control over the parameter assignments, you can also open
		the \emph{Organize Assignment Dialog} via the Flow Object's property sheet
		or context menu and manually create the individual Assignments.  Select
		a Property to assign the value to, e.g. one of the input parameters of
		the Web service's input message, and enter a from expression.
	\end{itemize}
	 To refer to a Property in the expression, just use the Property's name in an
	 expression, e.g. \verb_foo + 1_.  Note the assign time value: if this is set
	 to {\sc before}, the assignment will be made before the Activity is executed,
	 i.e. the Web service is invoked, otherwise the assignment will be made
	 afterwards.  Thus this value should be set to {\sc before}, when passing
	 values from the process to a Service's input parameter, and to {\sc after},
	 when passing values from a Service's output parameter back to the process.
\end{enumerate}


%%%%%%%%%%%%%%%%%%%%%%%%%%%%%%%%%%%%%%%%%%%%%%%%%%%%%%%%%%%%%%%%%%%%%%%%%%%%%%%%

\section{Validation and Simulation}

When your process is done -- or seems to be done -- you should validate it.  There
are several means for validation in the VSDT: First, you can validate the process
diagram against the constraints given in the BPMN specification; second, you can
check the structure of the process, which is important for most transformations
to executable code; and third, you can run a simulation, testing the several
Expressions, Conditions, and Assignments.

\begin{enumerate}

	\item To validate the diagram against the constraints from the BPMN
	specification, select \emph{Diagram $\rightarrow$ Validate} from the menu, or
	by clicking the checkmark symbol in the tool bar.  You might notice some error
	or warning marks in your diagram or entries in the problem view.  You should
	fix these problems before exporting the diagram to executable code.
	
	\item For checking the structure of the process, open the Structure View (see
	Section~\ref{sec:user_perspective_vsdtviews}) and click the \emph{Structurize}
	button.  This will trigger the same Structure Mapping used in the actual
	transformations and display the result, i.e. a structured form of the process,
	featuring elements such as sequences and blocks.  While the structured model
	might be a bit cumbersome to read, it gives evidence of the structure that
	will be recognized from the process, and if this is not the structure you
	intended you should consider restructuring the process.  Unfortunately, most
	executable languages are much more restrictive than process notations such as
	BPMN, so this check is necessary.
	
	\item For a more in-depth validation of the process -- especially the expressions
	used in assignments and conditions -- you may consider running a simulation.
	Currently there are two types of simulations implemented: a manual simulation
	and an interpreting simulation (see Section~\ref{sec:user_features_sim}).
	\begin{itemize}
		\item Use the manual simulation to get a feel of how the process behaves
		when taking a certain path, and to identify possible deadlock situations
		
		\item When using the VSDT Expression Language (VXL, see Section
		\ref{sec:user_features_exp}), the interpreting simulation will help you
		validating the several condition and assignment expressions used throughout
		the process. Make sure that the process is marked as being exectuable.
	\end{itemize}

\end{enumerate}


%%%%%%%%%%%%%%%%%%%%%%%%%%%%%%%%%%%%%%%%%%%%%%%%%%%%%%%%%%%%%%%%%%%%%%%%%%%%%%%%

\section{Code Generation}
\label{sec:user_tut_export}
	
Once all three validations are successful, the diagram can be translated to
executable code.  Of course, there might still be semantic errors in the process
the validation can not uncover, so you should think about thoroughly testing the
resulting program code before deploying it to productive use.

Before invoking the transformation to code, make sure that the diagram is marked
as being executable in the Business Process Diagram's properties sheet.
	
\begin{enumerate}

	\item Once the diagram shows no more errors, it can be exported to executable
	code.  Select \emph{Export...} from the file menu or from the model file's
	context menu.  Select the desired target language from the \emph{BPMN Export}
	group and proceed through the dialog.  Select the model file(s) to be exported,
	adjust the target directory or the other options, if necessary, and hit the
	\emph{Finish} button.
	
	\item Note that the Business Process System has an attribute which marks
	whether the system is intended to be executable (used for code generation) or
	not (used for documentation only).  This flag is used to determine e.g. which
	type of simulation is used by default, or how strict to validate the diagram.
	Before generating code, be sure to set this flat to executable, otherwise the
	transformation might abort with an error!
	
	\item The export might take some seconds.  If the model is sound, the output
	files will be created in a new directory in the specified target directory,
	named after the Business Process Diagram.  By default, also a log file will
	be created along with the model files in the specified target directory.
	
	\item If the process has been modeled accurately, the resulting program can
	be readily executable.  Still it is recommended to check the result with a
	native editor for the respective language, to be sure the files are free from
	defects.

\end{enumerate}

