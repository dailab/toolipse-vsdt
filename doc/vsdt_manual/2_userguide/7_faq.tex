\chapter{FAQ}
\label{sec:user_faq}

This chapter will give valuable advice on how to solve several tasks, solve problems, and answer other questions related to the Visual Service Design Tool.


\section{How to...}
\label{sec:user_faq_howto}

\paragraph*{\dots draw Flow Objects on the Canvas?}
Although sometimes such diagrams can be seen, Flow Objects \emph{can not} be drawn on the canvas. Instead, each Flow Object has to be contained in a Lane, and the Lane has to be inside of a Pool. However, according to the specification one Pool per diagram may have an invisible border, which can be set using the property sheets.

\paragraph*{\dots create an Embedded Sub Process?}
First, create a basic Activity. Then, in the Activity's property sheet, set the activity type to \textsc{Embedded Sub Process}. Now you can resize the Activity/Sub Process to an appropriate size and create further Flow Objects inside of it.

\paragraph*{\dots create an Intermediate Event on an Activity's boundary?}
For creating an Intermediate Event on an Activity's boundary, i.e. an event- or error handler, you should select the Event type from the palette and click the Activity's label or boundary, and \emph{not} the compartment. Alternatively, you can use the miniature palette that appears when hovering the mouse over the Activity's label or boundary. Once created, the Intermediate Event can be moved freely around the Activity's border.

\paragraph*{\dots draw Artifacts inside of a Pool?}
Contrary to Flow Objects, Artifacts can not be created inside of a Pool. However, you can create the Artifact on the canvas and drag it over the Pool. But remember that the Artifact is \emph{over}, and not \emph{in}, the Pool, so it will not be moved together with the Pool.

\paragraph*{\dots make Assignments to Message Parameters?}
The arguments and results of service invocations are being stored in the Properties of the input Message and the output Message.  So if a service is to be called with the argument "Hello World", that value has to be assigned to the respective Property of the service's input Message using the \emph{Organize Assignments Dialog} of the Activity calling the service.  Note, that Assignments to the input parameters have to be made \emph{before}, and Assignments storing the output parameters in local variables have to be made \emph{after} the Activities execution.  The Assign Time can be specified in the Dialog, too.  However, a more convenient way, and sufficient in most situations, is to use the \emph{Parameter Assignments Dialog}, which will take care of most of these details.  Of course, the same applies to Message Events and Send and Receive Activities, as well.

%\paragraph*{\dots enter a Time Date value?}
%This value is expected in quite complex form, because it has to be processed by a parser. The value has to be in the form "yyyy-MM-dd'T'HH:mm:ss.SSSZ", e.g. "2007-02-12T14:53:00.000+0100" for Monday, the 12th of February 2007 at 14:53:00 CET.


\section{Frequently Asked Questions}
\label{sec:user_faq_faq}

\paragraph{My diagram is broken. How can I fix it?}

In case the diagram is broken, it can be recreated from the model.  In case you are using separate files for semantic model and notational model, delete the diagram file \texttt{vsdt_diagram} and create a new one by right-clicking the model file \texttt{vsdt} and selecting the respective menu item.  In case you are using one file for both semantic and notational model, \texttt{vsdtd}, open the file in a text editor and remove the \texttt{notation:Diagram} elements (one for each diagram contained in the file). Now rename the file to \texttt{vsdt} and initialize the diagram file (in this case you can delete the model file afterwards, as the model has been copied to the diagram file).  In both cases, you will have to recreate the diagram's layout.


\section{Troubleshooting}
\label{sec:user_faq_trouble}

\paragraph*{I've drawn a legal Business Process Diagram, but when I try to export it the resulting code is broken.}
The diagram has to be in a certain form so it can successfully be transformed. The diagram has to be \emph{structured}. Primarily, there must not be blocks or loops with multiple entry or exit points. Although the VSDT can handle some forms of slight unstructuredness, for best results you should design your diagram in a way it can be mapped to a block-oriented language in a straight-forward manner.

\paragraph*{I've created a Business Process Diagram using element XYZ, but in the resulting code this element is missing.}
Not each single feature of BPMN can be taken into account in all of the transformations yet.  For those elements that can not be mapped, a no-operation element will be created in the target model, such as a \verb|empty| Activity in BPEL or a \verb|logwarn| in JIAC. Be sure to substitute these elements with a proper implementation after the transformation.
