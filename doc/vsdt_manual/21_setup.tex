\chapter{Setup}
\label{sec:user_setup}

In the following the installation process for the VSDT will be explained.  As the
VSDT is a plugin to the Eclipse IDE, the installation process is subdivided in
several steps:
\begin{enumerate}
	\item Install Eclipse
	\item Install Dependencies
	\item Install VSDT
\end{enumerate}

\emph{Note} that for running the VSDT Version 5 of the Java environment is required.

\emph{Note} that the versions of the dependencies may vary depending on the
version of the VSDT.  The following applies to version \version of the VSDT.


\section{Installing Eclipse}
\label{sec:user_setup_eclipse}

For using the Visual Service Design Tool you need the Eclipse IDE which can be
downloaded from \url{http://www.eclipse.org} for different operating systems.

The VSDT \version\ works with Eclipse 3.5 ``Galileo''.\footnote{For Eclipse 3.4
you can still use the VSDT up to version 1.2.2.}


\section{Installing Dependencies}
\label{sec:user_setup_dep}

The VSDT depends on a number of other plugins which again will have some
dependencies on its own.  It is recommended to use the Eclipse Update feature, as
in that case all the dependencies and the dependencies of the dependencies will
automatically be installed, too.  The dependencies are:

\begin{itemize}
	\item Graphical Modeling Framework SDK (group ``Modeling'')\footnote{The VSDT
	\version\ was created with GMF 2.1 and has been adapted to work with GMF 2.2.}
	
	\item XText SDK (group ``Modeling'')
\end{itemize}

For installing the plugins, make sure you are allowed to write in the Eclipse
program folder, especially when running Linux you should start that instance of
Eclipse with \texttt{sudo}.  In Eclipse, select \emph{Help} from the menu, then
\emph{Install New Software...}.  Select the \emph{Galileo} repository from the
drop-down menu and select the Dependencies from the list below.  Now click on
\emph{Next} to automatically check calculate further dependencies and continue
through the installation process.  After that you might have to restart Eclipse.
If the installation was successful the new features should appear in the
\emph{Installed Software} tab of the same dialog.

Theoretically, Eclipse can resolve the dependencies on its own, so it should be
enough to select the VSDT itself.  Still, we recommend to install the dependencies
manually.


\section{Installing the VSDT}
\label{sec:user_setup_vsdt}

Once Eclipse and GMF are set up you can install VSDT.  The Visual Service Design
Tool consists of the following features:

\begin{itemize}
	\item \emph{VSDT: Visual Service Design Tool} This is the core component of
	the VSDT, the visual BPMN editor, along with a number of useful extensions
	and the basic transformation framework.

%	\item \emph{VSDT: RSD Integration} This feature contributes an additional
%	view to the BPMN editor, making the reuse of existing Web services easier.

%	\item \emph{VSDT: Transformation Base} This is an enabling feature for the
%	several export features, providing the common functionality and look and
%	feel for the transformations.

	\item \emph{VSDT: BPMN - BPEL Transformation} This feature provides the export
	to executable BPEL code.

	\item \emph{VSDT: BPMN - JIAC V Transformation} This feature provides the
	export to JIAC V multiagent-systems.  Additional Dependency: \emph{Agent World
	Editor}

	\item \emph{VSDT: BPMN - STP-BPMN Transformation} This feature provides
	exchange with the popular Eclipse STP BPMN editor.  Additional Dependency:
	\emph{BPMN Project Feature} (

	\item \emph{VSDT: BPMN - \dots Transformation} Further export and import
	features will be released over time, providing transformations to other
	languages, such as JIAC agents.
\end{itemize}

The VSDT can be obtained from the following sites:
\begin{center}
	\downloadsites
\end{center}

Download the VSDT as an archived update site and add the site in the \emph{Install
New software} dialog (see last section), using the \emph{Add...} button.

