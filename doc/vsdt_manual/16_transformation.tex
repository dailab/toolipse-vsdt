\chapter{Model Transformation}
\label{sec:user_trafo}

The core of the Visual Service Design Tool clearly is the transformation to
executable code.  While by now the transformation to BPEL is the only one that
can be conveniently used in practice, there are currently several other
transformations under development.

For directions on how to invoke the transformation please refer to
Section~\ref{sec:user_tut_export}.


%%%%%%%%%%%%%%%%%%%%%%%%%%%%%%%%%%%%%%%%%%%%%%%%%%%%%%%%%%%%%%%%%%%%%%%%%%%%%%%%

\section{Understanding the Transformation Framework}
\label{sec:user_trafo_intro}

This section will provide a brief introduction in the basics of the transformation
framework.  The transformation framework has been designed from the very beginning
to be as \emph{extensible} and \emph{reusable} as possible.  For that purpose the
process of transformation has been subdivided into several stages, which are
sequentially applied to the input model:
\begin{enumerate}
	\item In the \emph{Validation} stage, all identifiers are validated to contain
	only characters that are legal with respect to the given target language.
	Further, the validation will check if each element needed is in place and
	providing clear error messages in case something is missing.
	
	\item The intent of the \emph{Normalisation} stage is to put the process
	diagram in a uniform form, and to transform it to a semantically equivalent
	representation of the diagram following more strict constraints than those
	given in the BPMN specification.
	
	\item In the \emph{Structure Mapping} stage, the model is searched for graph
	patterns which are semantically equivalent to block structures.  When such
	patterns are found, they are replaced with a special structured element, until
	the entire process within each Pool has been reduced to a single complex
	element, e.g.\ a sequence, or until it can not be reduced any further due to
	structural flaws.
	
	\item In the \emph{Element Mapping} stage, the several BPMN elements are
	mapped to their counterparts in the target language, e.g.  BPEL or JIAC
	(Section~\ref{sec:user_trafo_bpel} and~\ref{sec:user_trafo_jiac}).
	
	\item In the \emph{Clean Up} stage, a set of rules is applied on the newly
	created target model, improving the readability of the generated code and
	removing redundancies.
\end{enumerate}

A simple example of the consecutive execution of normalisation and structure
mapping can be seen in Figure~\ref{fig:norm_struc}.

\begin{figure}[ht]
	\centering
	\includegraphics[width=\textwidth]{figures/trafo/norm_struc.pdf}
	\caption{Simple example of normalisation and structure mapping.}
	\label{fig:norm_struc}
\end{figure}


\subsection{Transformation of Expressions}

Besides the actual workflow, Expressions that are used in assignments and
conditions have to be translated, too.  This can be done only if the Expression
Language is set to ``VSDT Expression Language'' or ``VXL''.\footnote{The Expression
Language can be set either globally in the Diagram properties or individually for
each Expression.} If then the \emph{Translate Expression} option is checked in
the Export Wizard (see Figure~\ref{fig:trafo_wiz}) these expressions will be
parsed and, if possible, translated to the respective target language.

\begin{figure}[ht]
	\centering
	\includegraphics[width=.5\textwidth]{figures/features/exportWiz.png}
	\caption{BPEL Export Wizard with Expression Translation checked.}
	\label{fig:trafo_wiz}
\end{figure}

Still there may be cases when VXL does not have enough expressive power.  In this
case the option can be disabled (or the Expression Language can be changed) and
the \emph{Replace Property Accessors} option can be checked.  In this case, the
Expressions will only be scanned for Property Accessors using VXL (e.g.
\texttt{\$foo.bar}) which will then be translated to the syntax of the target
language.  Thus, the simple VXL variables can be embedded in expressions of
another language.  For instance, in the case of BPEL, an expression like
\texttt{\$foo.bar + 1}, might be changed to \texttt{bpws:getVariableData(
'Proc\_ProcessData','foo','bar')+1}.  Thus the user does not have to care about
the way Properties are aggregated to variables in the transformation to that
language but can simply use a Property's name.

Finally, when Properties are given one of VXL's predefined basic data types (e.g.
\texttt{string}, \texttt{boolean}, etc.), where will be translated to the
respective basic types of the target language, e.g. \texttt{xsd:string} and
\texttt{xsd:boolean}.


%%%%%%%%%%%%%%%%%%%%%%%%%%%%%%%%%%%%%%%%%%%%%%%%%%%%%%%%%%%%%%%%%%%%%%%%%%%%%%%%

\section{Transformation Implementations}
\label{sec:user_trafo_impl}

The following sections describe in short the various transformations that have
already been implemented.


\subsection{Transformation to Text}
\label{sec:user_trafo_text}

See Section~\ref{sec:user_features_text}.


\subsection{Transformation to BPEL}
\label{sec:user_trafo_bpel}

The transformation to BPEL presented in this work covers nearly the entire mapping
as given in the BPMN specification~\cite[Appendix A]{omg2009bpmn}, including event
handlers, inclusive \textsc{or} and event-based \textsc{xor} Gateways, just to
name a few.

\paragraph{Export}
Nevertheless there are some elements for which the mapping is not given very
clearly, such as \textsc{timer} Start Events, independent Sub Processes or
multi-instance parallel loops.  While these elements will be transformed as
described in the specification, the resulting BPEL processes will require some
amount of manual refinement.  Besides the BPEL process files a WSDL definitions
file is created, holding the message types derived from the process properties
and the input and output messages and interfaces (port types) for the several Web
services being orchestrated by the process.  Still, the WSDL's binding and service
blocks and necessary schema types, if any, can not be generated automatically
yet, due to insufficient information in the source model.

In the validation, all identifiers are tested to contain only characters that are
legal with respect to BPEL, and all expressions used e.g.\ in Assignments and
loop conditions are translated to XPath, if possible.  Properties are aggregated
to one variable per Process or Message, for instance, if a Process \texttt{Proc}
has a Property \texttt{foo}, \texttt{foo} will be a Part of Variable
\texttt{Proc\_ProcessData}.

\paragraph{Import}
The Import from BPEL to BPMN is still at an early stage.  While basic control
flow can be imported, there are still problems e.g. with event- and fault handlers.
Further one should be aware that the export to BPEL does not preserve all
information in the BPMN diagram, thus diagrams re-imported after being exported
to BPEL will most likely be less readable than before, although they may be
semantically equivalent.


\subsection{Transformation to JIAC}
\label{sec:user_trafo_jiac}

Concerning our goal of transforming BPMN diagrams to multi-agent systems (MAS)
the work is still at an early stage.  First, a \emph{normal form} for BPMN diagrams
has been investigated, to facilitate the mapping~\cite{endert2007towards}.  Later,
the first steps of the actual mapping have been developed, basically mapping Pools
to agents, Processes and Flow Objects to the agents' plans and the control flow,
and Message Flow to the exchange of messages between the agents~\cite{endert2007mapping}.

\paragraph{Export}
A prototypic transformation targeting the agent framework JIAC V has already been
implemented.  In this transformation, the Participants diagram is translated to
an Agent World diagram, and the individual Business Processes are translated to
several JADL services.  As the theoretical part of the mapping is not yet fully
matured, there is still some work to do.  However, with the given transformation
framework every addition to the mapping can quickly be adopted.

The \emph{BPMN to JIAC Transformation} feature requires the \emph{JIAC Agent World
Editor} feature.


\subsection{Transformation to STP-BPMN}
\label{sec:user_trafo_stp}

Since BPMN does not provide a standardized interchange format, transferring a
diagram from one editor to another is difficult.  For that reason the VSDT provides
a Transformation to the popular Eclipse STP-BPMN Editor.  Thus one can export a
Diagram to STP-BPMN and carry it to a colleague using that editor, or someone
formerly using the STP editor can import his existing files when changing to the
VSDT.

\paragraph{Export}
The export to STP is nearly complete.  The only elements that are not mapped are
Lanes, which is because the STP editor is encountering problems when initializing
the diagram files containing Lanes.  Alas, the same also applies to Embedded
Subprocesses, as there are problems opening those diagrams.  However, this is a
problem with the STP editor that will get fixed soon.

\paragraph{Import}
The import from STP works very well, with the exception that, again, Lanes will
not be mapped. \emph{Note} that the transformation to STP will only create the
BPMN model files, but not the diagram files; those files, holding the graphical
information, have to be generated anew from the model file.  Thus, the transformation
does not preserve the layout of the diagrams.

The \emph{BPMN to STP-BPMN Transformation} feature requires the \emph{Eclipse SOA
Tools Project BPMN Editor} feature.


%%%%%%%%%%%%%%%%%%%%%%%%%%%%%%%%%%%%%%%%%%%%%%%%%%%%%%%%%%%%%%%%%%%%%%%%%%%%%%%%

\section{Limitations}
\label{sec:user_trafo_limits}

Although the transformation framework is quite powerful, it is important to
understand that there are still some limitations, both in the mapping of structures
and elements.

\begin{itemize}
	\item Given a well-structured workflow, the transformation will yield very
	good results.  However, as BPMN is more powerful that block-oriented languages,
	such as BPEL and JIAC, there will always be graph structures that can not be
	transformed to an equivalent program of the target language.  In that case
	the diagram should be manually restructured \emph{before} being exported.
	
	\item The mapping to BPEL is as complete as possible, according to the
	specification.  Still, there are some points in the specification that are
	missing, unclear or ambiguous.  The transformation to BPEL is implementing
	the specification in almost every detail, but regarding these elements a small
	amount of manual refinement may be necessary.
	
	\item The mapping to JIAC is still at an early stage.  JIAC models created
	with this transformation will require a large amount of manual work.
\end{itemize}

We are constantly investigating ways of extending both the quality of the element
mappings and the performance of the structure mapping.

