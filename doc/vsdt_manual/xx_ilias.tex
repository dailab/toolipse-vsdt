\chapter{Using the VSDT in ILIas}

% TODO am Ende noch andere Teile voruebergehend auskommentieren, die fuer ILIas
% weniger relevant sind (kommt da ueberhaupt wesentlich was zusammen?)
% - BPEL- und STP-Trafo-Zeugs (oder Trafo-Zeugs allgemein?)
% - JIAC-Node-Plugin?
% - Layout-Algorithmus (kann evtl. generell weg, oder in nen Anhang)

% Intro VSDT in ILIas (von Silvan, aus dem Meilenstein 2)
In order to create processes for managing the smart grid environment, a
corresponding development environment is adopted for better process creation and
-improvement.  One part of this environment is the process oriented development
of services for the SmaGriM system.  The main elements for this are services that
are provided as well defined functionalities by a service provider, in order to
allow service consumers to use those functionalities.  Such services can be
combined into processes that define concrete combinations of service calls, thus
realizing more complex functionalities.  Such processes are also called service
compositions and provide an easy to read yet powerful modeling approach for the
required grid management functionalities.

% Intro Prozess-Simulation (von Silvan, aus dem Meilenstein 2)
The ILIas system allows for extensive testing of services and processes developed
for the SmaGriM platform.  Simulations that are used to predict the behavior of
the grid to specific actions can also be used to test the effects of new or changed
implementations in what-if simulation scenarios.  Such scenarios are realized as
extended test cases by defining a test grid with known behavior, simulating new
services and processes in this scenario and comparing the simulation results to
the expected behavior of the grid.  This allows assessing the effects of changes
in a very concise procedure reducing overall testing efforts.

% Outline dieses Kapitels
Using the Visual Service Design Tool (VSDT), those processes can be modeled,
simulated (tested) using the SmaGriM simulation and finally be deployed to the
system.  In the following Sections, we will describe the general process of using
the VSDT in the ILIas project and then provide details about how to simulate the
process models prior to deploying them to the live system.


%%%%%%%%%%%%%%%%%%%%%%%%%%%%%%%%%%%%%%%%%%%%%%%%%%%%%%%%%%%%%%%%%%%%%%%%%%%%%%%%
%%  The ILIas Development Process                                             %%
%%%%%%%%%%%%%%%%%%%%%%%%%%%%%%%%%%%%%%%%%%%%%%%%%%%%%%%%%%%%%%%%%%%%%%%%%%%%%%%%

\section{The ILIas Development Process}

% TODO Development Process
% verschiedene Rollen und Entwicklungsphasen beschreiben
% Dienst-Provider: Basisdienste bereitstellen
% Prozess-Entwickler: Steuerungsprozesse erstellen und in Repository einchecken
% SmaGrim-Maintainer: Steuerungsprozess aus Repository auschecken, ggf. an eigene
%     Beduerfnisse anpassen, in Simulation durchspielen und auf Produktivsystem
%     deployen




%%%%%%%%%%%%%%%%%%%%%%%%%%%%%%%%%%%%%%%%%%%%%%%%%%%%%%%%%%%%%%%%%%%%%%%%%%%%%%%%
%%  Testing Processes in the ILIas Simulation                                 %%
%%%%%%%%%%%%%%%%%%%%%%%%%%%%%%%%%%%%%%%%%%%%%%%%%%%%%%%%%%%%%%%%%%%%%%%%%%%%%%%%

\section{Testing Processes in the ILIas Simulation}

% TODO ILIAS Simulation

\subsection{The ILIas Simulation View}
% Beschreibung der eigentlichen View, der einzelnen Buttons, etc.

\subsection{Simulating Processes}
% Klick-Weg zum Simulieren eines Prozesses:
% Welche Bedingungen muessen erfuellt sein, was muss alles eingestellt werden, etc.

\subsection{Deploying and Executing Processes}
% Nach der Simulation: Deployment auf 'Produktiv-Umgebung' (geht das auch ueber
% die Simulation View, oder verwendet man hierfuer wieder die Deployment View?)



