\begin{titlepage}
\begin{center}
\large \textbf{Thesis to obtain the Diploma Degree at the TU Berlin} \\
\vspace{2.0cm}
\Large \textbf{Development of a Visual Service Design Tool \\ providing a mapping from BPMN to JIAC} \\
\normalsize
\vspace{3.0cm}
Tobias Küster \\
\texttt{tobias.kuester@dai-labor.de}\\
\vspace{1.5cm}
\today \\
\vspace{3.0cm}
DAI Laboratory \\
Technical University of Berlin \\
Faculty IV -- Electrical Engineering and Computer Science \\
Prof. Dr. Sahin Albayrak \\
\end{center}
\end{titlepage}



% eidesstattliche erklärung
\vspace{2cm}

Die selbstständige und eigenhändige Anfertigung dieser Diplomarbeit versichere ich an Eides statt.

\vspace{2cm}

\begin{tabular}{p{5cm}p{5cm}}
Berlin,  & \\
\hline
        & \multicolumn{1}{c}{Tobias Küster}
\end{tabular}
\newpage



% zusammenfassungen
\section*{Abstract}

Although they have been topic to academic research for several years and although they are often seen as the next step towards truly reusable and modular programs, multi-agent systems are still underrepresented in the industry. At the same time web-services and service oriented architectures are adopted much faster in the same domain. A possible reason for this can be found in the lack of easy-to-use tools for the model driven creation of multi-agent systems.

In this thesis a tool for the model driven development of programs for the multi-agent system JIAC will be presented. For this purpose a mapping from business process diagrams based on the Business Process Modeling Notation to JIAC IV has been developed. Based on the Eclipse framework a visual editor for the creation of BPMN diagrams has been implemented. Diagrams created with this editor can be validated and translated to executable JIAC programs using a rule based transformation tool.



\section*{Zusammenfassung}

Obwohl sie seit vielen Jahren erforscht und oft als nächster Schritt in Richtung wiederverwendbarer, modularer Programme angesehen werden, sind Multi-Agenten-Systeme in der Industrie nach wie vor nur sehr schwach vertreten. Zur gleichen Zeit werden Web-Services und serviceorientierte Architekturen auf demselben Gebiet sehr viel besser angenommen. Eine mögliche Ursache hierfür kann in dem Fehlen von einfach zu handhabenden Werkzeugen für die modellgetriebene Erstellung von Multi-Agenten-Systemen gefunden werden.

In dieser Diplomarbeit wird ein Werkzeug für die modellgetriebene Entwicklung von Programmen für das Multi-Agenten-System JIAC vorgestellt. Für diesen Zweck wurde eine Abbildung von Geschäftsprozess-Diagrammen basierend auf der Business Process Modeling Notation nach JIAC IV entwickelt. Basierend auf der Eclipse-Technologie wurde ein visueller Editor für die Erstellung von BPMN-Diagrammen implementiert. Die mit dem Editor erstellten Diagramme können validiert und mithilfe eines regelbasierten Transformationswerkzeugs in ausführbare JIAC-Pro\-gramme übersetzt werden.

\newpage



% danksagung
\section*{Acknowledgment}

I want to thank Axel Heßler and Holger Endert for their support and suggestions. Further I want to thank the entire DAI Laboratory for providing me with a comfortable working environment and an interesting and challenging topic for my thesis.

Special thanks go to the TFS Group for providing us with the latest version of the EMT framework and to the Eclipse GMF newsgroup for answering all of my questions.

Finally I want to thank my whole family for their constant support.

\newpage