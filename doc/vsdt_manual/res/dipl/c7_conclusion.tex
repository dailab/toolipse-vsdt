\chapter{Conclusion}
\label{chapter:conclusion}

In this work a \textsc{Visual Service Design Tool} for the model driven development of programs for the multi-agent system JIAC has been developed. The tool is capable of transforming diagrams given in the Business Process Modeling Notation to JIAC and BPEL.

Firstly a survey on various workflow and business process notations has been made which lead to the decision to use BPMN as the source language for the mapping. After that we stated some of the goals and benefits of model driven engineering, followed by an introduction of the Eclipse Modeling Framework and the Graphical Modeling Framework. Further some approaches and known problems in the field of workflow transformation have been discussed, at which the main problem turned out to be the transformation of unstructured workflows to structured ones. For that purpose we introduced the rule-based graph transformation framework EMT.

We developed a mapping from BPMN to JIAC IV, which is concentrating on the control flow while the organizational structures, that is, the agents and platforms, are left out yet. We have discussed a number of approaches how BPMN diagrams could be mapped to agent concepts which are not covered by the original specification.

The visual editor for creating the business process diagrams has been developed using Eclipse GMF. Therefore the BPMN specification had to be implemented as an EMF Ecore model and mapped to nodes matching the notations given in the specification. Further, a number of constraints have been extracted from the specification and implemented as a model validation feature.

The mapping has been implemented as a transformation tool using both a top-down traversion and a rule-based transformation. The rules have been realized using the EMT framework. While well-structured diagrams can be transformed nicely there is still some work to do to support the transformation of unstructured workflows. The transformation tool has been integrated in the editor as an export wizard. Along with the mapping to JIAC also the mapping to BPEL that is proposed in the BPMN specification has been implemented.



\section{Future Work}

Future work to be done includes the review of some details of the mapping given in this thesis as well as the improvement of the \textsc{Visual Service Design Tool} and its integration in the larger context of the JIAC Toolsuite.


\subsection{Completion of the Mapping and JIAC TNG}

The most urgent work to be done is the completion of the mapping specification with respect to agents and agent platforms, which are not regarded in the current mapping yet. A number of possible approaches for the mapping of BPMN elements to agents has been given in section \ref{sec:jiacMapping} of which the introduction of new custom Artifacts representing agent concepts seemed to be the most promising one.

Further work has to be done extending the transformation rules to cover more, possibly all kinds of unstructured workflows.

Eventually the mapping will have to be ported to the upcoming JIAC TNG, replacing the current JIAC IV soon.


\subsection{Import and Round-Tripping}

Once the mapping from BPMN to JIAC is complete the direction from JIAC back to BPMN should be implemented. For this it would be favorable that both the mappings from BPMN to JIAC and from JIAC to BPMN were bijective. Otherwise it could not be granted that the consecutive execution of the mappings will results in the original diagram again, so that if $exp: BPMN \rightarrow JIAC$ and $imp:JIAC \rightarrow BPMN$ are the mappings then $imp \circ exp = id$ and $imp(exp(x))=x$.

Further, the export and import features together could be used to provide round-trip engineering, so that changes on the generated JIAC programs can be transferred back to the BPMN diagrams. In the case that the mappings can not be bijective round-trip engineering still could be provided by extending the reference model.


\subsection{Rich Client Platform and GMF 2.0}

As mentioned earlier GMF can be used to generate not only plugins but also self-contained applications, so called \emph{Rich Client Platform} (RCP) applications .

GMF is supporting the creation of RCP applications from version 2.0M4 on. The development platform, however, was GMF 2.0M2 which had no RCP support. Since the later milestone versions were not rid of all bugs we decided to wait for the GMF 2.0 final release scheduled for Friday June 29, 2007\footnote{\url{http://wiki.eclipse.org/index.php/GMF_Project_Plan}} instead of adapting the existing projects with each milestone release.

As soon as the final version of GMF 2.0 is available it would be reasonable to refactor or recreate the editor with this new version to benefit from all the new features. Also the \textsc{Visual Service Design Tool} should be created both as plugins and RCP application to reach as many interested parties as possible.


\subsection{Integration in the JIAC Toolsuite}

Finally the \textsc{Visual Service Design Tool} could become a part of the existing JIAC Toolsuite. Being integrated into the Toolipse environment, which is based on the Eclipse platform, too, the VSDT could contribute by providing easy, model driven generation of new JIAC services while at the same time it could benefit from the numerous features aiding the creation and maintenance of JIAC programs that are currently aggregated in the JIAC Toolsuite. Further, the resulting tool could be enriched with a repository feature for the organization and reutilization of existing business process diagrams.
