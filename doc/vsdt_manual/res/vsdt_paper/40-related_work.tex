\section{Related Work}
\label{sec:sota}

% grosse zahl von BPMN-Editoren, die meisten nur Zeichen-Tools ohne Export
The Business Process Modelling Notation has been adopted in a large number of tools.  Although many of these are merely diagram drawing tools and do not support the transformation to BPEL, let alone other languages, there are some powerful tools as well.  In the following we will introduce some of these.  A more extensive list can be found at \url{http://www.bpmn.org/BPMN_Supporters.htm}.

% Soyatec eBPMN und Eclarus SOA Architect
With the free \emph{eBPMN}, Soyatec provides a very nice BPMN editor, but it does not implement the mapping to BPEL.\footnote{\url{http://www.soyatec.com/ebpmn}}  The same applies to the free community edition of eClarus' \emph{Business Process Modeller}, while the commercial \emph{SOA-Architect} version provides a transformation to BPEL, although it seems to have some limitations.\footnote{\url{http://www.eclarus.com/}}
% Intalio / Eclipse STP
A very mature BPM product can be found in the \emph{Intalio BPMS}.\footnote{\url{http://www.intalio.com/}}  BPEL code is generated on-the-fly and can be deployed to the Intalio process engine.  However, the mapping of workflow structures is limited, e.g.\ we found it impossible to merge a branch originating from an event handler back into the normal flow.  Another limitation arises from the tight coupling to the in-house BPEL engine, which is using some proprietary extensions.  Further, Intalio has donated parts of the code to the Eclipse SOA Tools Project (STP).  While the \emph{STP BPMN Editor} itself does not provide a transformation to BPEL, Giner et.al.\ were able to combine it with the \emph{BABEL Bpmn2Bpel} tool~\cite{giner2007bridging}, yet both the editor and the transformation tool are using very simple metamodels.


% Mapping von Unstructured Workflows
Concerning the transformation from graph-oriented to block-oriented process models, as in the BPMN to BPEL case, Mendling et.~al.\ have evaluated several transformation strategies~\cite{mendling2005transformation}, ranging from a straight-forward mapping of BPMN Sequence Flows to BPEL Links, similar to the one in~\cite{white2005using}, to a more sophisticated \emph{Structure Identification} strategy, like the one applied in this work, or \emph{Structure Maximisation}, as followed by Aalst and Lassen~\cite{aalst2008translating}.  In their theoretically well-founded, pattern-based transformation from Petri nets to BPEL, they focus on the readability of the resulting code.  However, they do not regard how highly \emph{unstructured} workflows can be transformed to structured ones.  As pointed out in~\cite{recker2006translation}, there is a ``mismatch'' between BPMN and BPEL, both on the domain representation and the control flow level, that is not easily to overcome.  Many authors have investigated whether different graph patterns can be transformed to an equivalent structured form~\cite{Kiepuszewski2000structured, liu2005Analysis, sadiq2000analyzing}, and came to the conclusion that even slight unstructuredness can require the introduction of additional variables or the duplication of parts of the workflow, even though the workflow models used in these works are much simpler than BPMN.  For structuring such workflows, Koehler et.~al.\ present a rule-based transformation based on continuation semantics~\cite{koehler2004untangling}.  Another approach is followed by Ouyang et.~al., using BPEL event handlers as a form of \texttt{goto} command~\cite{ouyang2006translating_standard}.  Their examples show how complicated a simple workflow can become when being structured.

Thus, as workflow design will be facilitated greatly if the user is not restricted to the use of block-oriented processes, a transformation of unstructured workflows to readily executable code will be highly desirable, so that such processes can be created by means of Model Driven Engineering.
