\section{Introduction}
\label{sec:intro}

% Einleitung:
% - Prozessmodellierung sehr beliebt in der Geschaeftswelt, MDE
% - kurze Einleitung in BPMN
% - soll heterogene Systeme ermoeglichen
% - Unabhaengigkeit des Modells von einer konkreten Implementierungssprache, etwa BPEL
The goal of process modelling, as of Model Driven Engineering in general, is to provide an abstract view on systems, and to design those systems in a language and platform independent way.  For that purpose the Business Process Modelling Notation (BPMN)~\cite{omg2006business} has been standardised by the Object Management Group.  It can be understood intuitively by all business partners, even those who have great knowledge in their domain but do not know too much about Service Oriented Architecture (SOA) or programming in general.  At the same time, BPMN is formal enough to provide a basis for the later implementation and refinement of the business process.  Given a respective mapping, a BPMN diagram can be used for generating readily executable code from it.  A brief introduction to BPMN is given for instance in~\cite{white2004introduction}.

% Problem:
% - die meisten Tools bieten nur eine Trafo nach BPEL an
% - durch proprietaere Metamodelle lassen sich die Modelle nicht in einem anderen Tools verwenden
% - keine Unabhaengigkeit von der Implementierungssprache BPEL mehr
Today, the Business Process Modelling Notation and the specified mapping to the Business Process Execution Language (BPEL) are supported by a growing number of tools --- we will have a closer look on some representatives later in Section~\ref{sec:sota}.  However, the problem with the majority of existing tools is that while they do provide the usual transformations from BPMN to BPEL, they are focused only on this one aspect of BPMN.  Often the editors and even the underlying metamodels are adapted to BPEL in many ways.  While this may be desired in order to provide highest possible usability and to support the user in the creation of executable BPEL code, the consequence is that business process diagrams created with these tools can neither be transformed to other executable languages, nor can the process model be used with other tools that might provide different transformations.  Thus, while process modelling and BPMN should be independent of a specific executable language, the \emph{tools} are not.

% Loesung:
% - alles Implementierungsspezifische wird aus dem Metamodell und dem Editor herausgehalten
% - mehrere Transformationen in unterschiedliche Zielsprachen koennen als Plugins hinzugefuegt werden
The solution to this problem is to keep both the underlying BPMN metamodel and the diagram editor free from influences from the BPEL world and to use pure BPMN instead, so that diagrams created with such a tool will be truly independent of any concrete language --- apart from what influenced the BPMN specification in the first place.  Based on this, several mappings to different target languages can be implemented and integrated into the editor as plugins, which may also contribute to the editor in order to support the business architect with language-specific support.

% Contribution:
% - ein Tool, mit dem Business Prozesse in verschiedene Sprachen transformiert werden koennen
% - Trafo nach BPEL als Proof-of-Concept fertiggestellt
% - Implementierungsaufwand der Plugins durch modulare Architektur verringert
% - weitere Transformationen nach Agentensystemen in der Mache
% - geplant: Transformation in heterogene Systeme
Following this approach, the \emph{Visual Service Design Tool} (VSDT) has been implemented as an Eclipse plugin, inherently providing the necessary modularity, as we will see in Section~\ref{sec:editor}.  For the export of BPMN diagrams to executable languages a transformation framework has been designed, which we will describe in more detail in Section~\ref{sec:trafo}.  The transformation has been subdivided in distinct stages, so that significant parts of it are reusable, e.g.\ the challenging transformation of the control flow.  Thus the actual mapping to a given language can be integrated in a very straight-forward way.  While the usual mapping from BPMN to BPEL has been realised as a proof of concept (see Section~\ref{sec:trafo_bpel}), the main intent behind the VSDT is to provide a transformation from business processes to multi-agent systems such as the JIAC language family~\cite{sesseler2002modularearchitektur}.  The respective mappings are currently under development and will be discussed briefly in Section~\ref{sec:trafo_jiac}.  Our ultimate goal is to provide transformations not only in different, but also in heterogeneous systems --- just like they are used in the real business world.  Future work in order to achieve this goal, as well as a conclusion to this paper, will be discussed in Section~\ref{sec:conclusion}.
