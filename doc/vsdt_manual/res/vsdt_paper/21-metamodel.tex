\subsection{The Metamodel}
\label{sec:editor_meta}

% Ueber die BPMN Spezifikation
The BPMN specification~\cite{omg2006business} describes in detail how the several nodes and connections constituting a BPMN diagram have to look, in which context they may be used and what attributes they have to provide.  However, it does neither give a formal definition of the syntax to be used for the metamodel, nor an interchange format, e.g.\ using an XML Schema Definition (XSD).  Thus the editor's metamodel had to be derived from the informal descriptions in the specification.
% Detaillierte Umsetzung der Spezifikation im Modell
As it was our main concern to keep as close to the specification as possible, we decided not to reuse the existing Eclipse STP BPMN Editor, which uses a simplified model of BPMN.\footnote{\url{http://www.eclipse.org/stp/bpmn/}}  Instead, almost every attribute and each constraint given in the specification has been incorporated into the metamodel, allowing the creation of any legal business process diagram.
% Sachen, wo sich das Modell von der Spezifikation unterscheidet
Still, some attributes have not been adopted in the metamodel:  For instance the possibility to model nested or even crossing Lanes has been dropped, as it turned out that this feature seems to be virtually never used in practical business process design.  Further, redundant attributes, such as the Gateway's \texttt{defaultGate} attribute, are emulated using getter methods to prevent inconsistency in the diagram model.

% StructuredBPMN-Metamodell
Concerning the transformation to BPEL and other executable languages, which in most cases are block-oriented, an extension to the usual BPMN metamodel has been designed, featuring equivalents to the basic block structures, such as sequences, decisions, parallel blocks and loops.  These elements are described in a separate metamodel, extending the editor's metamodel.  They are used only during the transformation process, especially for the mapping of the structure, as we will see in Section~\ref{sec:trafo}.
