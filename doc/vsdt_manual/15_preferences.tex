\chapter{Preferences}
\label{sec:user_preferences}

This chapter will explain the several preferences that can be set for configuring
the VSDT for your personal needs.

The preference pages can be accessed by navigating to \emph{Window $\rightarrow$
Preferences\dots} in the menu and selecting the item \emph{VSDT} from the list.

The GMF framework provides a number of settings on its own, e.g. for improving
performance by deactivating certain features of the GMF runtime, or for setting
the default colors and fonts for certain diagram elements.  As these settings are
self-explaining we will not go further into detail on those.  In the following we
will explain the VSDT-specific settings only.

\paragraph{General}
\begin{itemize}
	\item \textbf{Author} sets the default author for all new Business Process
	Diagrams created.
	
	\item \textbf{Enable Modeling Assistance} turns GMF's modeling assistance on
	or off.
\end{itemize}	

\paragraph{Appearance}
\begin{itemize}
	\item \textbf{Enable Activity Icons} activates additional markers placed in
	the corners of Activities.  These markers use intuitive symbols to indicate
	the Activity's type and whether the Activity has any Assignments and/or
	Properties.
	
	\item \textbf{Use Additional Colors} helps to distinguish the several diagram
	elements by the use of colors.
	
	\item \textbf{Show XOR-Marker for Gateways} sets whether to display the
	cross-marker for XOR gateways.
	
	\item \textbf{Meta Diagram Style} lets you choose between a notation similar
	to UML use cases or BPMN 2.0 communication diagrams for the VSDT's entry
	diagrams.
\end{itemize}

\paragraph{Connections}
Here you can select the drawing style for Sequence Flows, Message Flows and
Associations.

\emph{Note} that both the Enable Activity Icons and the Use Additional Colors
settings take effect only after closing and reopening a given Business Process
Diagram file.

