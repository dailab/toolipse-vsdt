\chapter{The VSDT Expression Language (VXL)}
\label{sec:vxl}

% Vorstellung VXL
The BPMN standard does not specify an expression language to be used.  Instead,
it is assumed that the language of the target framework is used, e.g.  XPath.
However, in a tool that provides transformations to various target frameworks
this is not an option.  While the diagram structure could be translated to the
syntax of the target system, the expression, given that they are written in an
unknown language, could not -- although all those languages might be very similar.
To address this flaw, the VSDT comes with its own, very simple expression language,
the \emph{VSDT Expression Language (VXL)}.


%%%%%%%%%%%%%%%%%%%%%%%%%%%%%%%%%%%%%%%%%%%%%%%%%%%%%%%%%%%%%%%%%%%%%%%%%%%%%%%%

\section{Language Features and Syntax}

The VSDT Expression language has been designed to be the greatest common divisor
of the expression languages used in the target frameworks.  Thus, most expressions
can be given using VXL, in which case they can be validated and -- more importantly
-- parsed and translated to the respective expression languages used in the target
frameworks.

Below, the complete syntax of VXL is given.  As can be seen, it is not much
different from that of other languages.  Variables have to be the name of a
Property in the scope of the owner of the Expression.  The Variable may be followed
by one or more accessors, e.g. for access to an array element (e.g. \verb_foo[2]_)
or a field (e.g. \verb_foo.bar_), or combinations thereof (e.g. \verb_foo[n+1].bar_);
of course accessors can only be used if the target language and data type supports
them.

\begin{verbatim}
Term:           Element (Operator Term)?
Element:        BracketTerm | Negation | Minus | Card |
                List | Map | Variable | Value
BracketTerm:    "(" Term ")"
Negation:       "not" Element
Minus:          "-" Element
Card:           "#" Element
List:           "[" ListElement? "]"
ListElement:    Term ("," ListElement)?
Map:            "{" MapElement? "}"
MapElement:     key = Term ":" value = Term ("," MapElement)?
Variable:       ID Accessor?
Accessor:       ArrayAccessor | FieldAccessor
ArrayAccessor:  "[" Term "]" Accessor?
FieldAccessor:  "." ID Accessor?
Value:          STRING | INT | FLOAT| "true" | "false" | "null"
Operator:       "<" | "<=" | "==" | "!=" | ">" | ">=" | 
                "+" | "-" | "*" | "/" | "%" | "and" | "or" | "++"
\end{verbatim}


%%%%%%%%%%%%%%%%%%%%%%%%%%%%%%%%%%%%%%%%%%%%%%%%%%%%%%%%%%%%%%%%%%%%%%%%%%%%%%%%

\section{Operators and Comparisons}

In Table~\ref{tab:vxl_op} we spell out the several operations and comparisons
supported in VXL, most of which can be found in all the usual programming languages
and do not require further explanation.

\begin{table}[htbp]
\centering
	\caption{VXL Operations and Comparisons}
	\begin{tabular}{lc||lc}
		\hline
		\bf Operation   & \bf Symbol & \bf Comparison   & \bf Symbol \\
		\hline
		Addition        & \verb_+_	 & Equal            & \verb_==_  \\
		Subtraction     & \verb_-_	 & Not Equal        & \verb_!=_  \\
		Multiplication  & \verb_*_   & Lesser           & \verb_<_   \\
		Division        & \verb_/_   & Lesser or Equal  & \verb_<=_  \\
		Modulo          & \verb_%_   & Greater          & \verb_>_   \\
		Concatenation   & \verb_++_  & Greater or Equal & \verb_>=_  \\
		Cardinality     & \verb_#_   &                  &            \\
		Logical \sc and & \verb_and_ &                  &            \\
		Logical \sc or  & \verb_or_  &                  &            \\
		Logical \sc not & \verb_not_ &                  &            \\
		\hline
	\end{tabular}
	\label{tab:vxl_op}
\end{table}

The \verb_==_ comparison should test whether two values are \emph{equal}, but not
necessarily the \emph{same}, and the semantics of \verb_!=_ should be understood
accordingly.  The unary Cardinality operation can be used for retrieving the length
of a list or a string, for getting the absolute value of a number, or for casting
a boolean value to either $1$ or $0$.  The Concatenation operation is used to
combine lists or strings, such that \verb_[1,2]++[3,4]_ results in \verb_[1,2,3,4]_,
whereas \verb_[1,2]+[3,4]_ results in \verb_[1,2,[3,4]]_.  The \verb_*_ operator
can be used to construct lists, e.g. \verb_[0]*5_ results in \verb_[0,0,0,0,0]_

Note, that this is how these operators \emph{should} be understood, and how the
VSDT Interpreter interprets evaluates them.  However, it is possible that parts
of the language are not supported, or interpreted differently, in some transformation
target languages.

