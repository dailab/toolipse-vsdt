\chapter{Setup}
\label{sec:user_setup}

In the following the VSDT's installation process will be explained.  As the VSDT
is an Eclipse plug-in, the installation process consists of three steps:
\begin{enumerate}
	\item Install Eclipse
	\item Install Dependencies
	\item Install VSDT
\end{enumerate}

\emph{Note} that for running the VSDT \emph{Java} is required in \emph{Version 5
or higher}.


%%%%%%%%%%%%%%%%%%%%%%%%%%%%%%%%%%%%%%%%%%%%%%%%%%%%%%%%%%%%%%%%%%%%%%%%%%%%%%%%

\section{Installing Eclipse}

For using the Visual Service Design Tool you need the Eclipse IDE which can be
downloaded from \url{http://www.eclipse.org} for different operating systems.

The current version of the VSDT, \version, works with Eclipse 4.2 ``Juno''.
For Eclipse 3.5, you can use the VSDT in version 1.4.0, and for Eclipse 3.4, you
can still use the VSDT up to version 1.2.2.




%%%%%%%%%%%%%%%%%%%%%%%%%%%%%%%%%%%%%%%%%%%%%%%%%%%%%%%%%%%%%%%%%%%%%%%%%%%%%%%%

\section{Installing Dependencies}

The VSDT depends on a number of other plug-ins which again will have some
dependencies on their own.  It is recommended to use the Eclipse Update feature,
since this way all the dependencies and the dependencies of the dependencies
will automatically be installed, too.  The dependencies are:

\begin{itemize}
	\item Graphical Modeling Framework SDK (group ``Modeling'')\footnote{The VSDT
	\version\ was created with GMF 2.1 and has been adapted to work with later
	Versions of GMF}

	\item XText SDK (group ``Modeling'')
\end{itemize}

% aktuell fuer Eclipse-Version 3.5
For installing the plug-ins, make sure you are allowed to write in the Eclipse
program folder; especially when running Linux you should start that instance of
Eclipse with \texttt{sudo}.  In Eclipse, select \emph{Help} from the menu, then
\emph{Install New Software...}.  Select the \emph{Galileo} repository from the
drop-down menu and select the Dependencies from the list below.  Now click on
\emph{Next} to automatically calculate further dependencies (this can take a few
moments) and continue through the installation process.  After that you might
have to restart Eclipse.  If the installation was successful the new features
should appear in the \emph{Installed Software} tab of the same dialog.

Theoretically, Eclipse can resolve the dependencies on its own, so it should be
enough to select the VSDT itself.  Still, we recommend to install the dependencies
manually.


%%%%%%%%%%%%%%%%%%%%%%%%%%%%%%%%%%%%%%%%%%%%%%%%%%%%%%%%%%%%%%%%%%%%%%%%%%%%%%%%

\section{Installing the VSDT}

Once Eclipse and GMF are set up you can install the Visual Service Design Tool
itself. The latest version of the VSDT can be obtained from the following site(s):
\begin{center}
	\downloadsites
\end{center}

Download the VSDT as an archived update site and add the site in the \emph{Install
New software} dialog (see last section), using the \emph{Add...} button.

The VSDT's update site consists of the following features:

\begin{itemize}
	\item \emph{VSDT: Visual Service Design Tool} This is the core component of
	the VSDT, the visual BPMN editor, along with a number of useful extensions
	and the basic transformation framework.

%	\item \emph{VSDT: RSD Integration} This feature contributes an additional
%	view to the BPMN editor, making the reuse of existing Web services easier.

%	\item \emph{VSDT: Transformation Base} This is an enabling feature for the
%	several export features, providing the common functionality and look and
%	feel for the transformations.

	\item \emph{VSDT: BPMN - BPEL Transformation} This feature provides the export
	to executable BPEL code.

	\item \emph{VSDT: BPMN - JIAC V Transformation} This feature provides the
	export to JIAC V multi-agent systems.  Additional Dependency: \emph{Agent World
	Editor}

% 	\item \emph{VSDT: BPMN - STP-BPMN Transformation} This feature provides
% 	exchange with the popular Eclipse STP BPMN editor.  Additional Dependency:
% 	\emph{BPMN Project Feature}

	\item \emph{Agent World Editor} An modeling tool for creating and configuring
	JIAC agents.  Required for the transformation from BPMN to JIAC V.

	\item \emph{JIAC Utils} A JIAC node running inside an Eclipse plug-in,
	providing some useful functionality, being described in
	Chapter~\ref{sec:user_jiac-node}.
\end{itemize}

