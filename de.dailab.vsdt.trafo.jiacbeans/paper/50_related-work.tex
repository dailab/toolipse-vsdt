%%%%%%%%%%%%%%%%%%%%%%%%%%%%%%%%%%%%%%%%%%%%%%%%%%%%%%%%%%%%%%%%%%%%%%%%%%%%%%%%
%%                                                                            %%
%%  RELATED WORK                                                              %%
%%                                                                            %%
%%%%%%%%%%%%%%%%%%%%%%%%%%%%%%%%%%%%%%%%%%%%%%%%%%%%%%%%%%%%%%%%%%%%%%%%%%%%%%%%

% [x] Intro
% [x] BPMN -> BPEL
% [x] BPMN -> JADL
% [x] WADE
% [x] GO-BPMN

\section{Related Work}
\label{sec:related}

In the following we will discuss several works which are highly relevant to the
approach described in this paper: The original mapping from BPMN to BPEL, a
mapping from BPMN to JIAC's scripting language JADL, the WADE framework, mapping
workflows to JADE behaviours, and GO-BPMN, a combination of BPMN and goal hierarchies.


%%%%%%%%%%%%%%%%%%%%%%%%%%%%%%%%%%%%%%%%%%%%%%%%%%%%%%%%%%%%%%%%%%%%%%%%%%%%%%%%

\subsection{Mapping BPMN to BPEL}

One of the motivations for developing BPMN was to provide a standardised graphical
notation for \emph{BPEL}, the Business Process Executable Language.  Consequently,
a mapping from BPMN to BPEL is part of the BPMN specification~\cite[Appendix
A]{omg2009bpmn}, and a number of alternative or extended mappings have been
proposed by various other authors (see for example~\cite{ouyang2009business}).
% , making heavy use of BPEL event handlers).

In many aspects, the mapping is very straightforward, as it is evident that BPMN
was created with the mapping to BPEL in mind: Each Pool is mapped to a BPEL
process (which can be deployed as a Web service), and the several events and
activities within are mapped to the workflow of the process.  The process is made
up mostly of Web service calls, assignments and flow control, but can also contain
e.g. event handling based on timing and incoming messages.  Given a sufficiently
detailed BPMN diagram, the resulting BPEL process can be readily executable.

Still, there are enough elements in BPMN for which no mapping to BPEL is given,
i.e. BPMN is not just a visualization for BPEL but an individual language -- and
in fact more expressive than BPEL itself.  Amongst the elements which are not
mapped to BPEL are somewhat esoteric elements such as the \emph{ad-hoc} subprocess,
or the complex gateway, but also many kinds of events and tasks.


%%%%%%%%%%%%%%%%%%%%%%%%%%%%%%%%%%%%%%%%%%%%%%%%%%%%%%%%%%%%%%%%%%%%%%%%%%%%%%%%

\subsection{Mapping BPMN to JADL}

% erste Version des Mapping nach JADL gemacht, SOA+Agenten-Sprache
In prior work of mapping BPMN to JIAC agents, JIAC's service-oriented scripting
language \emph{JADL}~\cite{hirsch2010programming} was selected as the target of
the transformation.

% Behaviours werden in JADL beschrieben
Being conceptually close to BPEL, the mapping is similar, and the process can be
mapped very directly to different language elements of JADL -- for instance, like
BPEL, JADL has dedicated language elements for complex actions such as invoking
another service, or for sending and receiving messages, making the generated code
compact and easy to comprehend.  Similar to the mapping to BPEL, each Pool in the
BPMN process is mapped to a JADL service, and the service's input parameters and
result types are derived from the Pool's start- and end
events~\cite{kuester2010integrating}.

% Starter-Regeln mit Drools
Further, for each start event, a Drools rule is created, starting the respective
JADL service on the occurrence of the given event (e.g. an incoming message, or
a given time).

% Agenten-Konfiguration
Finally, for each Participant in the BPMN process, an XML-based agent configuration
file is created, setting up the individual agents, each equipped with an Interpreter
Bean and Rule Engine Bean, together with the generated JADL services and Drools
rules.  Alternatively, the JADL services created from the BPMN processes can be
deployed to a running JIAC agent, thus dynamically changing its behaviour.


%%%%%%%%%%%%%%%%%%%%%%%%%%%%%%%%%%%%%%%%%%%%%%%%%%%%%%%%%%%%%%%%%%%%%%%%%%%%%%%%

\subsection{WADE: Workflows for JADE}

% WADE: JADE + Workflows
A different approach, from which some of the concepts in this work have been
drawn, is \emph{WADE (Workflows and Agents Development Environment)}, which is
an extension to the JADE multi-agent framework~\cite{bellifemine1999jade}.
With WADE, certain aspects of the behaviour of a JADE agent can be modelled
using a simple workflow notation~\cite{caire2008wade,caire2010wade}, based on
which Java classes are generated.  The workflows basically consist of only two
elements: Activities and Transitions.

% generiert Java-Code, pro Aktivität ein Methodenrumpf, Workflow-Inheritance
Using the \emph{Wolf} tool~\cite{caire2008wolf}, JADE behaviour classes can be
generated from those workflow models.  In the generated Java classes, there is
a clear distinction between the workflow (the order of the activities, together
with conditions and guards, where required) and the several activities, each of
which is mapped to an individual Java method, which can either refer to existing
functionalities or be implemented by the developer.  Using this separation,
generated workflows can easily be altered or extended.

% Schlecht: sehr einfache Workflow-Notationen, nichts agentisches
However, the expressiveness of WADE is restricted by the simplistic workflow
model, which allows only the most basic workflows to be modelled.  While the
transitions can be annotated with guards (conditions), it seems impossible to
model parallel execution and synchronisation, let alone more advanced concepts
such as event handling or messaging.  In fact, each workflow diagram covers only
the behaviour of an isolated agent; to our knowledge, interactions between agents
can not be modelled.


%%%%%%%%%%%%%%%%%%%%%%%%%%%%%%%%%%%%%%%%%%%%%%%%%%%%%%%%%%%%%%%%%%%%%%%%%%%%%%%%

\subsection{GO-BPMN}
% XXX wenn noetig, hier kuerzen (Intro nicht vergessen)

% Worum es bei GO-BPMN geht: Zielhierarchien und Prozesse, Kritik der Autoren
In another approach, \emph{GO-BPMN} (Goal-oriented BPMN), process models are
combined with a goal-hierarchy and executed by agents~\cite{calisti2008goaloriented}.
The authors praise the high flexibility of the system, and the prospects of
parallelisation, but they also write that testing the system is difficult due to
possible side-effects of the processes regarding other goals~\cite{burmeister2008bdiagents}.

% eigene Kritik am Ansatz: BPMN zu low-level, nur einzelne Prozesse, keine Kommunikation
As the name implies, the individual processes (the ``leafs'' in the goal hierarchy)
are described as BPMN processes.  However, only a subset of BPMN is used.
Particularly, each diagram shows only a single Pool, and thus, as in the case of
WADE, no communication and interaction can be modelled, but just the behaviour of
a single agent.  While using goals for connecting the individual processes is
quite promising, in our opinion process diagrams can more efficiently be used at
a higher level of abstraction, e.g. for providing an overview of the system as a
whole, instead of for isolated behaviours of individual agents.

