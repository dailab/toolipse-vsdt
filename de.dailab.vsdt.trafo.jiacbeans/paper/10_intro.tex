%%%%%%%%%%%%%%%%%%%%%%%%%%%%%%%%%%%%%%%%%%%%%%%%%%%%%%%%%%%%%%%%%%%%%%%%%%%%%%%%
%%                                                                            %%
%%  INTRODUCTION                                                              %%
%%                                                                            %%
%%%%%%%%%%%%%%%%%%%%%%%%%%%%%%%%%%%%%%%%%%%%%%%%%%%%%%%%%%%%%%%%%%%%%%%%%%%%%%%%

% [x] Problem
% [x] Motivation
% [x] Solution
% [x] Contribution
% [x] Outline

\section{Introduction}
\label{sec:intro}

% Problem
In recent times, different approaches have been introduced, using business
process diagrams or related notations for modelling agents and multi-agent
systems~\cite{caire2008wade,kuester2010integrating}.  However, none of these
approaches is really compelling.  Often, very simple workflow models are used, or
if a more expressive process modelling notation, such as BPMN~\cite{omg2009bpmn}
is used, only a limited subset of the language is covered.  Further, usually only
single agents are targeted, while interactions between agents -- which could very
well be modelled using e.g. BPMN -- are not regarded.

% Motivation
This is a pity, since process diagrams -- particularly complex notations such as
the Business Process Modeling Notation (BPMN) -- share many concepts and abstractions
with multi-agent systems.  Besides the actual workflow, those notations can be
used for modelling several participants in a process, as well as their interactions
and communication, or their reactions to external events, centring much more on
the \emph{what} and less on the \emph{how}.  Thus, despite the shortcomings of
existing approaches, BPMN and related notations appear to be very well suited for
modelling agents and particularly multi-agent systems.

% Solution
In this paper we take a look at some of the existing approaches -- the WADE
extension to the JADE agent framework, and the mapping from BPMN to JIAC's
scripting language JADL -- and try to combine the strong sides of both into a new
approach.  The result is a mapping from BPMN to JIAC Agent Beans, following the
basic principles of WADE, but using a much more expressive process notation.

% Contribution
This way, JIAC Agent Beans, being the core components of the agents, can be
generated easily from BPMN processes.  The resulting Java classes are similarly
structured and extensible as the JADE classes of WADE, but reflect the whole BPMN
process, including communication between agents and event-handling, both as part
of the workflow and for triggering the process.


%%%%%%%%%%%%%%%%%%%%%%%%%%%%%%%%%%%%%%%%%%%%%%%%%%%%%%%%%%%%%%%%%%%%%%%%%%%%%%%%

% Outline
The remainder of this paper is structured as follows: In Section~\ref{sec:background}
we will give a short introduction to both, the BPMN process modelling language,
and the JIAC multi-agent framework.  Then, in Section~\ref{sec:related} we will
have a look at related work, most notably the WADE framework and the mapping from
BPMN to JADL.  Thereafter, we will describe in detail how BPMN processes can be
mapped to semantically equivalent JIAC Agent Beans (Section \ref{sec:mapping}),
and how the transformation has been implemented (Section~\ref{sec:impl}).
In Section~\ref{sec:example}, the mapping is illustrated using an example,
before we finally discuss our results in Section~\ref{sec:conclusion}.

