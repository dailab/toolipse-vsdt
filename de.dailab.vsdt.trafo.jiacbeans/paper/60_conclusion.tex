%%%%%%%%%%%%%%%%%%%%%%%%%%%%%%%%%%%%%%%%%%%%%%%%%%%%%%%%%%%%%%%%%%%%%%%%%%%%%%%%
%%                                                                            %%
%%  CONCLUSION                                                                %%
%%                                                                            %%
%%%%%%%%%%%%%%%%%%%%%%%%%%%%%%%%%%%%%%%%%%%%%%%%%%%%%%%%%%%%%%%%%%%%%%%%%%%%%%%%

% [x] Zusammenfassung
% [x] Diskussion
% [x] Future Work

\section{Conclusion}
\label{sec:conclusion}

% kurze Zusammenfassung des Papers und des Mappings
In this paper, we have presented a new approach of creating multi-agent systems from
process models, combining the mapping from BPMN to JADL~\cite{kuester2010integrating}
with ideas borrowed from WADE~\cite{caire2008wade}.  The result is a transformation
from BPMN process diagrams to JIAC Agent Beans, comprising one method for the
workflow as a whole, and one method for each individual activity, but also
supporting beneficial aspects of BPMN such as messaging and event handling.


%%%%%%%%%%%%%%%%%%%%%%%%%%%%%%%%%%%%%%%%%%%%%%%%%%%%%%%%%%%%%%%%%%%%%%%%%%%%%%%%

\subsection{Discussion}

% alles ausdrückbar, was man modellieren kann
Using the domain-specific scripting language JADL, agent behaviours can be
expressed in a very compact and readable way, but the overall expressiveness is
limited.  JIAC Agent Beans, on the other hand, have the full expressiveness of
the Java language at their disposal.  Thus, basically everything that can be
modelled in a BPMN diagram can be mapped to an Agent Bean.

% etwas schwer verständlich, erweiterbar
While the resulting workflow-methods for complex processes can become somewhat
bulky -- particularly if event handlers are used -- the separation into
workflow-methods and activity-methods keeps the resulting code reasonably clear.
Further, like in WADE, individual activity-methods can be altered or extended
without risk of losing the changes after the code is generated anew.

% Schlusssatz: Wofür eignen sich die AgentBeans, und wofür die JADL-Services?
One drawback compared to the mapping to JADL scripts is that the generated Agent
Beans -- being Java classes -- can not as easily be deployed to an agent at runtime.
Regarding the high expressiveness of the generated Agent Beans and the good
performance of compiled Java code compared to the interpreted JADL scripts, the
mapping from BPMN to JIAC Agent Beans is suited best for modelling and generating
core components of the agents, while the mapping to JADL is of much use for
creating dynamic behaviours and services to be deployed and changed at runtime.


%%%%%%%%%%%%%%%%%%%%%%%%%%%%%%%%%%%%%%%%%%%%%%%%%%%%%%%%%%%%%%%%%%%%%%%%%%%%%%%%

\subsection{Future Work}

While the mapping can already be used for generating useful agent behaviours,
it is not yet completed.  First, there are still aspects of BPMN, which are not
covered by the mapping, such as some of the less common event types.  Second,
there are aspects of agents that can not yet be modelled adequately with BPMN.

Among the latter are messages to individual agents.  Originally, BPMN knew only
services and service calls.  In the course of our work, we extended our `dialect'
of BPMN with broadcast message groups, but what's still missing are messages to
individual agents, as identified by their agent ID.

Another issue which we want to tackle in the future is the modelling of goals by
means of BPMN.  One approach that appears to be promising is to use the
\emph{ad-hoc} subprocess for this task, but this is still work in progress.

