%%% TITELSEITE %%%

\thispagestyle{empty}

\begin{center}

	\includegraphics[scale=0.5]{images/TuLogo.png}
	\hfill
	\includegraphics[scale=0.7]{images/DAI_Logo.png}

	\vspace{1cm}
	\textbf{\ARTDERARBEIT} \\

	\vspace{2cm}
	\textbf{\LARGE \TITELA}\\
	\vspace{0.2cm}
	\textbf{\LARGE \TITELB}

	\vspace{2cm}
	Fachbereich \emph{Agententechnologien in betrieblichen Anwendungen\\
	und der Telekommunikation} (\emph{AOT})\\
	\vspace{0.5cm}
	Prof.\ Dr.-Ing.\ habil.\ Sahin Albayrak \\
	Fakult\"{a}t IV Elektrotechnik und Informatik \\
	Technische Universit\"{a}t Berlin \\

	\vspace{2cm}
	Vorgelegt von: \textbf{\AUTOR} \\
	\vspace{0.5cm}
	Betreuer: \BETREUER \\

\end{center}

\vfill
\AUTOR \\
Matrikelnummer:  \MATRIKEL \\
\ADRESSE \\

\newpage


%%% EIDESSTATTLICHE ERKLAERUNG %%%

Die selbstst\"{a}ndige und eigenh\"{a}ndige Anfertigung dieser \ARTDERARBEIT\
versichere ich an Eides statt.

\vspace{4cm}
\parbox{6cm}{\hrule \strut \centering \small Berlin, 4. November 2011}
\hfill
\parbox{6cm}{\hrule \strut \centering \small \AUTOR}

\newpage


%%% ZUSAMMENFASSUNG %%%

\section*{Abstract}
% TODO Zusammenfassung auf Englisch (nur bei englischssprachiger Arbeit)
Models have been a solution to bridge the communication gap between the business users and the IT world, a common problem in the software industry. In the domain of multi agent systems, this communication gap is believed to be one of the reason why the agent concept is not very popular in the industry, although they have been a research topic for many years, while webservices and service oriented architectures that addresses a similar problem domain are adapted much faster by the business users. 
 
At the moment, some model driven approach has been made in order to simplify the development of multi agent systems providing tools which allows graphical process editing and a mapping of the process into agents. One example is the VSDT which provides a BPMN editor and a transformation framework and a mapping of BPMN into JIAC agents, allowing agents to be designed and created using BPMN. In this work we will present a mapping and transformation of BPMN into Java code or JIAC Agent Beans to be more specific by extending the VSDT's transformation framework.


\section*{Zusammenfassung}
% TODO Zusammenfassung auf Deutsch (in jedem Fall)
Die Kommunikationsl"ucke zwischen Unternehmen und IT ist ein weitverbreitetes Problem in der Softwareindustrie. Modelle (meistens bestehen aus graphische Zeichnungen) sind Mittel um diese L"ucke zu "uberbr"ucken. Im Bereich der Multi-Agenten-Systeme glaubt man, dass diese L"ucke eine der Ursache ist, weswegen das Agentenkonzept, obwohl es seit mehreren Jahren ein Forschungsthema ist, nicht sehr popul"ar ist in der Industrie, w"ahrend Konzepte wie Webservices und serviceorientierte Architekturen, die ein "ahnliches Problembereich anspricht, viel schneller angenommen wird von den Unternehmen.

Zur Zeit bestehen bereits einige Ans"atze um die Entwicklung von Multi-Agenten-Systeme zu vereinfachen. Diese Ans"atze stellt Werkzeuge bereit, die eine graphische Prozessmodellbearbeitung und die Abbildung der Prozesse zu Agenten erm"oglicht. Ein Beispiel daf"ur ist das VSDT, das ein BPMN Editor, ein Transformationsframework und eine Abbildung von BPMN zu JIAC Agenten bereitstellt und somit die Generierung von Agenten aus BPMN erm"oglicht.

\newpage


%%% DANKSAGUNGEN %%%

\section*{Acknowledgement}
I would like to thank Tobias K"uster for being a very good supervisor, for all the help, suggestions and motivation he gave me during the last 6 months.
I also want to thank my whole family for their constant love and support. 
% TODO Optional: Danksagung


