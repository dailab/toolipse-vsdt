%%% TITELSEITE %%%

\thispagestyle{empty}

\begin{center}

	\includegraphics[scale=0.5]{images/TuLogo.png}
	\hfill
	\includegraphics[scale=0.7]{images/DAI_Logo.png}

	\vspace{1cm}
	\textbf{\ARTDERARBEIT} \\

	\vspace{2cm}
	\textbf{\LARGE \TITELA}\\
	\vspace{0.2cm}
	\textbf{\LARGE \TITELB}

	\vspace{2cm}
	Fachbereich \emph{Agententechnologien in betrieblichen Anwendungen\\
	und der Telekommunikation} (\emph{AOT})\\
	\vspace{0.5cm}
	Prof.\ Dr.-Ing.\ habil.\ Sahin Albayrak \\
	Fakult\"{a}t IV Elektrotechnik und Informatik \\
	Technische Universit\"{a}t Berlin \\

	\vspace{2cm}
	Vorgelegt von: \textbf{\AUTOR} \\
	\vspace{0.5cm}
	Betreuer: \BETREUER \\

\end{center}

\vfill
\AUTOR \\
Matrikelnummer:  \MATRIKEL \\
\ADRESSE \\

\newpage


%%% EIDESSTATTLICHE ERKLAERUNG %%%

Die selbstst\"{a}ndige und eigenh\"{a}ndige Anfertigung dieser \ARTDERARBEIT\
versichere ich an Eides statt.

\vspace{4cm}
\parbox{6cm}{\hrule \strut \centering \small Berlin, 4. November 2011}
\hfill
\parbox{6cm}{\hrule \strut \centering \small \AUTOR}

\newpage


%%% ZUSAMMENFASSUNG %%%

\section*{Abstract}
% TODO Zusammenfassung auf Englisch (nur bei englischssprachiger Arbeit)
Models have been used as a solution to bridge the communication gap between the business users and the IT world, a common problem in the software industry. In the domain of multi agent systems, this communication gap is believed to be one of the reason why the agent concept is not very popular in the industry, although it has been a research topic for many years. On the other hand, webservices and service oriented architectures that addresses a similar problem domain are adapted much faster by the business users. 
 
At the moment, some model driven approach has been made in order to simplify the development of multi agent systems providing tools which allows graphical process editing and a mapping of the process into agents. One example is the VSDT which provides a BPMN editor, a transformation framework and a mapping of BPMN into JIAC agents, allowing agents to be designed and created using BPMN. In this work we will extend the VSDT's transformation framework in order to develop a transformation of BPMN into Java code or to be more specific, into JIAC Agent Beans.


\section*{Zusammenfassung}
% TODO Zusammenfassung auf Deutsch (in jedem Fall)
Die Kommunikationsl"ucke zwischen Unternehmen und IT stellt ein weitverbreitetes Problem in der Softwareindustrie dar. Als L"osung dieses Problems greift man oft auf Modelle zu. Auf dem Gebiet der Multi-Agenten-Systeme wird diese L"ucke f"ur eine Ursache gehalten, weswegen das Agentenkonzept, obwohl es seit mehreren Jahren ein Forschungsthema ist, in der Industrie nicht sehr popul"ar ist. Im Gegesatz dazu kommen bei den Unternehmen Konzepte wie Webservices und serviceorientierte Architekturen, die einen "ahnlichen Problembereich ansprechen, viel besser an.

Zur Zeit bestehen bereits einige Ans"atze um die Entwicklung von Multi-Agenten-Systeme zu vereinfachen. Diese Ans"atze stellen Werkzeuge bereit, die eine graphische Prozessmodellbearbeitung und die Abbildung der Prozesse zu den Agenten erm"oglichen. Ein Beispiel daf"ur ist Visual Service Design Tool (VSDT), das einen BPMN Editor, ein Transformationsframework und eine Abbildung von BPMN auf JIAC Agenten bereitstellt und dadurch die Generierung von Agenten aus BPMN erm"oglicht. In dieser Arbeit wird das Transformationsframework des VSDT erweitert, um eine Transformation von BPMN auf Javacode bzw. JIAC Agent Beans zu entwickeln. 

\newpage


%%% DANKSAGUNGEN %%%

\section*{Acknowledgement}
I would like to thank Tobias K"uster for being a very good supervisor, for all the help, suggestions and motivation he gave me during the last 6 months.

I also want to thank my whole family for their constant love and support. 
% TODO Optional: Danksagung


