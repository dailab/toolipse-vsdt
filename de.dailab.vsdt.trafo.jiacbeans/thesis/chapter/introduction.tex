\chapter{Introduction}
In this chapter, we will start by introducing the motivation and the goals of this work. 
\section{Motivation}
\label{sec:Motivation}
\subsection{Model Driven Engineering (MDE)}
Over the past few years more and more software developers have been adopting the principle of \textit{Model Driven Engineering}(MDE) where 
they no longer focus on writing programs but on creating a set of models which define the software. By modelling the software, the developer creates documents that provides an abstract view of the software system, independantly from the plattform or a specific programming language, making it understandable for non experts i.e. the stakeholders as well as applicable in different plattforms. These models will then be the basis for the implementation. A significant number of the so called  CASE (Computer Aided Software Engineering) tools have been developed to support this methodology. Beside providing support in creating and editing the models, most of these CASE tools are also equipped with transformation features that allows us to transform the model into text or even executeable Programs, thus increasing efficiency in the software development process. We can say that the real benefits of MDE lies in the transformation. By providing a mapping between the model and the code, we can create standardized programs, accelerate development time and minimize faults in writing the code. \\\\

\subsection{MDE in Multi-agent systems}
Back in 2007, an MDE-approach has been made in order to bridge the gap between the industry and the multi-agent systems. As a result, a CASE-tool called the \textbf{\textit{Visual Service Design Tool (VSDT)}} was developed by Tobias K\"uster in scope of his Diploma Thesis, which provides the transformation of BPMN (Business Process Modelling Notation) to BPEL(\textit{Business Process Execution Language}) and JIAC (Java-based Intelligent Agent Componentware) framework. This tool allows agents to be designed using a business process diagram, a model which has already been manifested in the industry. In the scope of this work, a plugin to VSDT will be developed to enrich it's transformation feature with a code generator that will transform BPMN models into executeable Java Code, or JIAC AgentBeans to be more specific. 

\section{Goals}
\label{sec:Goals}
The main goal of this work is to develop an eclipse plugin as an extension to VSDT to enrich its transformation features with a new transformation from BPMN to Java Code or JIAC AgentBeans to be more specific. Because it is nearly impossible to put all implementation  details into the model, and to anticipate the possibility, that the generated code will be edited manually, considerations has to be made, such that conflicts should not occur when the transformation is called to a code that has been edited manually. 


