\chapter{State of the art}

%====================================================%
%                     BPMN2BPEL                      %
%====================================================%
\section{BPMN to BPEL}
In the BPMN Specification, the mapping from BPMN to BPEL has been included since the first version. In fact the development of BPMN was driven by the lack of standard notation for the WS-BPEL\cite{weidlich2008}. 


For this Project the specified mapping from BPMN2BPEL are used to gain a better understanding of the notation's semantics.


%====================================================%
%                     BPMN2Agents                    %
%====================================================%
%======== Existing Transformation to Jiac ===========%
\section{Existing Transformation to JiacV}
At the moment, the VSDT is already equipped with a transformation of BPMN to JiacV, which generates the JADL(Jiac Agent Description Language).
The transformation to Jiac AgentBeans is in no way a replacement to the existing transformation. Instead both transformations shall complement each other as the products of both transformations have their own advantages, some of which we can find in the following table:
\begin{table}[htbp]
	\centering
		\begin{tabularx}{\linewidth}{|l|X|}\hline\hline
			\multicolumn{2}{|c|}{\textbf{Advantages of:}} \\\hline
			\multicolumn{1}{|c|}{JADL} & \multicolumn{1}{c|}{AgentBeans}\\\hline
			can be deployed to a running Jiac application &  Written fully in Java, therefore developer friendly.\\
																										&  Java is more powerful that JADL.\\
			                           										&  Better performance because no parser are involved.\\\hline\hline
		\end{tabularx}
\end{table}

\section{WADE}
A similar approach (designing agents behavior with processes and transforming it into Java Code) has been developed by the Telecom Italia with their JADE-extension called WADE (Workflows and Agents Development Environment). While Jade was developed to simplify the implementation of Software Agents, WADE extended JADE with a workflow engine, making it possible to create Agents that executes tasks defined as workflows.

\subsection{JADE (\textbf{J}ava \textbf{A}gent \textbf{DE}velopment Frameork)}
JADE is an application framework and a middleware written in Java, which support the development of software agents. The framework  provides distributed runtime environments, agent and behaviour abstractions as well as communications between agents and discovery mechanisms. We can say that it's role is very similar to JIAC.

\subsection{WADE (Workflows and Agents Development Environment)}
WADE is an extension to JADE, which enrich the application framework with a workflow engine. The WorkflowEngineAgent extends the JADE basic Agent class with an ability to execute workflows represented in a WADE specific formalism.

A WADE workflow is actually a Java Class, thus it can be edited and managed as Java classes and can contain pieces of code which is needed to implement the process. With WOLF, a development environment that comes with WADE, developers can edit the Workflow graphically as well as textually. The code view and the graphical view of the workflow are kept in sync. 


\subsection{Wade-Workflow}
A Process in Wade consists a set of Activities. A Process is defined with exactly one \textbf{Start Activity}, and one or multiple \textbf{End Activities} 
% explain the Wade-Workflow, add screnshots etc.
\\\\
Despite having all the advantages of a Java code, the WADE workflow is rather simple and not so expressive as the BPMN. Similarity to the agent technology such as events and communication flow are also missing in the workflow. 