\chapter{State of the art}

A similar approach (designing agents behaviour with processes and transforming it into Java Code) has been developed by the Telecom Italia with their JADE-extension called WADE (Workflows and Agents Development Environment). While Jade was developed to simplify the implementation of Software Agents, WADE extended JADE with a workflow engine, making it possible to create Agents that executes tasks defined as workflows.

\section{JADE (\textbf{J}ava \textbf{A}gent \textbf{DE}velopment Frameork)}
JADE is an application framework and a middleware written in Java, which support the development of software agents. The framework  provides distributed runtime environments, agent and behaviour abstractions as well as communications between agents and discovery mechanisms. We can say that it's role is very similar to JIAC.

\section{WADE (Workflows and Agents Development Environment)}
WADE \cite{16,10,12} is a software plattform built on top of JADE. One of the key component of WADE is the WorkflowEngineAgent which extends the basic Agent class of the JADE library with an ability to execute workflows represented in a WADE specific formalism.
A WADE Workflow is actually a Java Class, thus it can be edited and managed as Java classes and can contain pieces of code which is needed to implement the process. With WOLF, a development environment that comes with WADE, developers can edit the Workflow graphically as well as textually. The code view and the graphical view of the workflow are kept in sync. 
Despite having all the advantages of a Java code, the WADE workflow is rather simple and not so expressive as the BPMN.