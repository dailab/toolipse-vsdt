\chapter{Mapping BPMN to Jiac AgentBeans}

The element mapping from BPMN to Jiac AgentBeans is created based on the existing mapping to JiacV - JADL script. In comparation to a JADL script, an AgentBean is written completely in Java, enabling more possibilities in mapping concepts such as intermediate Event handling. This chapter will provide a tabular overview of the mapping. 

\textbf{\large{Business Process Diagram}}

		\begin{tabularx}{\linewidth}{|l|X|}\hline\hline
			\multicolumn{2}{|c|}{\textbf{Business Process Diagram}} \\\hline\hline
			 Pool & A pool is currently mapped to a Java-File containing a Jiac AgentBean.\\\hline
			 Process & The process content in a pool is mapped to a Workflow Method in the generated AgentBean. Depending on the type of the StartEvent, an Action might be exposed, enabling this method to be invoked as a service. This method will contain a Script which is generated from the flowObjects contained in the process.\\\hline
			 Lane & Lanes will not be mapped in this mapping\\\hline \hline
		\end{tabularx}\\\\

\textbf{\Large{Events}}\\\\
For intermediate Events, an EventHandlerThread will be generated and started. If the intermediate Event is attached to an activity, the activity will also be started as a Thread. 

		\begin{tabularx}{\linewidth}{|l|X|}\hline\hline
			\multicolumn{2}{|c|}{\textbf{Start Events}} \\\hline\hline
			 Timer & The generated process will be started in the execute() method of the JIAC AgentBean\\\hline
			 Message & If the implementation is a Service, an action will be created, and the process can be started through a service call (invoke). If the implementation is a Message Channel, a Space Observer will be attached to the agents memory, and the process will start as soon as a message is recieved.\\\hline
			 Rule & Rule start events are not regarded in the current mapping\\\hline
			 Link & \\\hline
			 Multiple & If a pool has multiple start events, each start event will be mapped according to it's respective trigger\\\hline\hline
			 
			 \multicolumn{2}{|c|}{\textbf{Intermediate Events}} \\\hline\hline
			 Rule & not regarded in the current mapping\\\hline
			 Timer & A TimerEventHandler will be created and started.\\\hline
			 Message & A MessageEventHandler will be created and started. \\\hline
			 Link & ...\\\hline
			 Multiple & each event will be mapped according to it's respective trigger\\\hline
			 Cancel & ... \\\hline
			 Compensate & ... \\\hline
			 Error & If attached to an activity, the method generated from the activity will be called in a try-catch block \\\hline\hline
			 
			 \multicolumn{2}{|c|}{\textbf{End Events}} \\\hline\hline
			 Message & If implementation is Message Channel, a message with the payload will be written to the channel \\\hline
			 Link & \\\hline
			 Multiple & \\\hline
			 Cancel & \\\hline
			 Compensate & \\\hline
			 Error & The generated service will throw an exception\\\hline
			 Terminate & \\\hline
		\end{tabularx}\\\\

\textbf{\large{Activities}}\\\\
Basically, an Activity-method will be created for each activity.  

\begin{table*}[h]
	\centering
		\begin{tabularx}{\linewidth}{|l|X|}\hline\hline
			\multicolumn{2}{|c|}{\textbf{Activity}} \\\hline\hline
			 Standard Loop & ... \\\hline
			 Multi Instance Loop & ... \\\hline
			 With Event Handler & the method generated will be started parallel to an EventHandlerThread. When the activity is completed, the EventHandlerThread will be terminated. In contrary, if the Event is trigerred before the activity is completed, the Thread performing the activity will be interrupted. \\\hline
		\end{tabularx}
\end{table*}\\\\